\newpage

\begin{center}
	\textbf{Лабораторная работа № 4}
\end{center}

Переписать приложение из предыдущей лабораторной работы с использованием следующих технологий:
\begin{itemize}
\item Уровень back-end должен быть основан на Spring.
\item Уровень front-end должен быть построен на Angular 2+ с использованием набора компонентов PrimeNG
\item Взаимодействие между уровнями back-end и front-end должно быть организовано посредством REST API.
\end{itemize}

Приложение по-прежнему должно включать в себя 2 страницы - стартовую и основную страницу приложения. Обе страницы приложения должны быть адаптированы для отображения в 3 режимах:
\begin{itemize}
\item "Десктопный" - для устройств, ширина экрана которых равна или превышает 1045 пикселей.
\item "Планшетный" - для устройств, ширина экрана которых равна или превышает 758, но меньше 1045 пикселей.
\item "Мобильный"- для устройств, ширина экрана которых меньше 758 пикселей.
\end{itemize}

Стартовая страница должна содержать следующие элементы:
\begin{itemize}
\item "Шапку", содержащую ФИО студента, номер группы и номер варианта.
\item Форму для ввода логина и пароля. Информация о зарегистрированных в системе пользователях должна храниться в отдельной таблице БД (пароль должен храниться в виде хэш-суммы). Доступ неавторизованных пользователей к основной странице приложения должен быть запрещён.
\end{itemize}

Основная страница приложения должна содержать следующие элементы:
\begin{itemize}
\item Набор полей ввода для задания координат точки и радиуса области в соответствии с вариантом задания: Dropdown \{ '-4', '-3', '-2', '-1', '0', '1', '2', '3', '4' \} для координаты по оси X, Text (-5 ... 3) для координаты по оси Y, и Dropdown \{ '-4', '-3', '-2', '-1', '0', '1', '2', '3', '4' \} для задания радиуса области. Если поле ввода допускает ввод заведомо некорректных данных (таких, например, как буквы в координатах точки или отрицательный радиус), то приложение должно осуществлять их валидацию.
\item Динамически обновляемую картинку, изображающую область на координатной плоскости в соответствии с номером варианта и точки, координаты которых были заданы пользователем. Клик по картинке должен инициировать сценарий, осуществляющий определение координат новой точки и отправку их на сервер для проверки её попадания в область. Цвет точек должен зависить от факта попадания / непопадания в область. Смена радиуса также должна инициировать перерисовку картинки.
\item Таблицу со списком результатов предыдущих проверок.
\item Кнопку, по которой аутентифицированный пользователь может закрыть свою сессию и вернуться на стартовую страницу приложения.
\end{itemize}

Дополнительные требования к приложению:
\begin{itemize}
\item Все результаты проверки должны сохраняться в базе данных под управлением СУБД Oracle.
\item Для доступа к БД необходимо использовать Spring Data.
\end{itemize}