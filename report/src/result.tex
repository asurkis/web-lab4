\newpage
\paragraph{Вывод:}

В данной лабораторной работе я решал задачу реализации веб-приложения, разделенного на две части:
серверной, ответственной за работу с базой данных,
и клиентской, взаимодействующей с серверной с помощью REST API\@.

Во время выполнения лабораторной я столкнулся со следующими проблемами:
\begin{itemize}
    \item Создание адаптивного интерфейса с продвинутыми компонентами -- для решения использовал
    Angular -- фреймворк, позволяющий разбить код клиентской части на компоненты и предоставляющий
    высокоуровневую обертку над браузерными API,
    PrimeFaces (PrimeNG) -- библиотеку готовых компонентов
    и PrimeFlex -- компонент FlexGrid, позволяющий создать адаптивную верстку.
    \item Неудобство написания кода на ,,голом'' JavaScript из-за только динамической типизации --
    вместо этого я использовал TypeScript, позволяющий писать как статически типизированный код,
    так и динамически типизированный (нужный при преобразовании JSON структуры во внутреннее представление).

    \item Обработка запросов на сервере -- использовал Spring Framework,
    позволяющий задать маппинги запросов в виде аннотаций, а ответы задавать в виде любого сериализуемого в JSON
    (в моем случае) типа.
    \item Идентификация пользователей -- использовал модуль Spring Session, автоматически создающий HTTP-сессию,
    по умолчанию на основе HTTP Cookie (что я и использовал), но настраиваемую и на другие варианты
    (например, заголовок \texttt{X-Auth-Token})
    \item Авторизация пользователей -- использовал модуль Spring Security, перехватывающий все запросы и проверяющий
    авторизацию, а также позволяющий задать правила определения авторизации пользователя и его прав.
    \item Валидация запросов на сервере -- использовал модуль Spring Form Validation
    \item Хранение данных -- использовал модуль Spring Data, предоставляющий абстракцию над базами данных и
    автоматически создающий реализацию метода обращения к БД по его сигнатуре.
    \item Генерацию REST-ответов -- использовал модуль Spring HATEOAS, позволяющий легко задать HAL-структуру с
    нужными ссылками.

    \item Громоздкость кода этой лабораторной на Java -- вместо этого использовал Kotlin, позволяющий сократить
    описание POJO-классов и добавляющий много ,,синтаксического сахара'', как методы-расширения, однострочные методы и т.д.
\end{itemize}
